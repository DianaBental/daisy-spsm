\documentclass[a4paper,10pt]{article}
\usepackage[utf8x]{inputenc}

%opening
\title{Query Repair}
\author{Andriana Gkaniatsou}

\begin{document}

\maketitle


\section{Introduction}

The repair process takes as input the mismatched query (both data and schema) and the SPSM matching results and produces new queries which are considered 
schematically correct. The repair process considers the following cases:

\begin{itemize}
 \item The query dataset name can match to more than one dataset names (we own).
\item Each dataset match (predicate match) comes with the corresponding argument matches.
\item Within the same dataset, an argument can have more than one matching (more than one arguments that can be mappes to).
\item A dataset match might correspond to more than one new queries.
\item A dataset macth that does not have any argument matching is discarded.
\end{itemize}

\section{Repair Algorithm}

First step is to repair the query schema according to the SPSM results. 
\begin{itemize}
 \item Given a query dataset name and the correspoing arguments, find all SPMS results that match the query dataset name to 
a dataset name that we own. Keep only the dataset name matches that produce argument matches.
\item For each $DatasetMatch_i$, where $QueryDataset \mapsto DatasetMatch_i$ find all corresponding $SingleArgumentMatches$
    \begin{itemize}
    \item For each $QA_i$ where $QA_i$ $\in$ QueryArguments, find all $QA_i$ that appear  one $SingleArgumentMatch$. 
     \item Find all  $QMA_i$ $\in$ QueryArguments, that have more than one $SingleArgumentMatches$, where  $SingleArgumentMatches= \{ SingleArgumentMatch_1, \\SingleArgumentMatch_2, .., SingleArgumentMatch_n\}$ and 
$QueryDataset(QA_i) \mapsto DatasetMatch_i(SingleArgumentMatch_1)$, \\
$QueryDataset(QA_i) \mapsto DatasetMatch_i(SingleArgumentMatch_2)$.., and $n \leq 2$
    \begin{itemize}
\item  For each $QMA_i$ select randomly $SingleArgumentMatches_i$ and replace it

    \begin{itemize}
     \item Replace each $QA_i$ with the corresponding  $SingleArgumentMatch$
      \item For each of the remaining $QA_i$ replace it with a randomly chosen  $SingleArgumentMatch_i$ 
      \item Delete all query arguments that do not have a corresponding $SingleArgumentMatch$
    \end{itemize}
\item Repeat for all $SingleArgumentMatches_i$
\end{itemize}

 \item Repeat untill all possible argument cobinations have been created. 


    \end{itemize}



\end{itemize}





To form the final query, we have to find all data values contained in the original query and change their type. 
Hence, for each query schema we have formed, we find (if any) the corresponding data values that are provided by the query.


\section{Forming the final query - Data Level Repair}

Inputs of this process are all query schemata that have been formed by the previous process and for each schema the corresponding data.
For each schema we form the corresponding query and we send it to the appropriate dataset. If no answers are returned, assuming that the schema is 
correct, we  uninstantiate the data values of the query.  Goal of this process is to identify the data values that cause the query failure. 
The steps we perform are the following:

\begin{itemize}
 \item Find all data values of the query
    \begin{itemize}
    \item Select one randomly and uninstantiate it
   \item Send query to the dataset again
    \item Check if it returned any answers
	\begin{itemize}
	
	\item If no answers were returned, pick another data value, and uninstantiate it 
	  \item Send the new query to the dataset
	  \item Check for answers
	\end{itemize}
      \item Repeat for all data values
      \end{itemize}

    \item Uninstantiate all data values of the query
    \item Send uninstantiated query to the appropriate dataset
      
\end{itemize}


\end{document}